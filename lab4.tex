% ****** Start of file aipsamp.tex ******
%
%   This file is part of the AIP files in the AIP distribution for REVTeX 4.
%   Version 4.1 of REVTeX, October 2009
%
%   Copyright (c) 2009 American Institute of Physics.
%
%   See the AIP README file for restrictions and more information.
%
% TeX'ing this file requires that you have AMS-LaTeX 2.0 installed
% as well as the rest of the prerequisites for REVTeX 4.1
% 
% It also requires running BibTeX. The commands are as follows:
%
%  1)  latex  aipsamp
%  2)  bibtex aipsamp
%  3)  latex  aipsamp
%  4)  latex  aipsamp
%
% Use this file as a source of example code for your aip document.
% Use the file aiptemplate.tex as a template for your document.
\documentclass[%
 aip,
 % onecolumn,
% jmp,
% bmf,
% sd,
 % rsi,
 amsmath,amssymb,
%preprint,%
 reprint,%
%author-year,%
%author-numerical,%
% Conference Proceedings
floatfix,
% tikz,
]{revtex4-1}

\usepackage{graphicx}% Include figure files
\usepackage[utf8]{inputenc}
\usepackage[T1]{fontenc}
\usepackage{mathptmx}
\usepackage{tikz}
\usetikzlibrary{arrows,decorations.markings,decorations.pathmorphing, patterns,shapes}
\usepackage{dcolumn}% Align table columns on decimal point
\usepackage{bm}% bold math
\usepackage{float}
%\usepackage[mathlines]{lineno}% Enable numbering of text and display math
%\linenumbers\relax % Commence numbering lines

\begin{document}
\preprint{AIP/123-QED}

\title[]{The Lifetime of the Muon}
% Force line breaks with \\

\author{Jared Baur and Ben Sappey}
 % \altaffiliation[Also at ]{Physics Department, XYZ University.}%Lines break automatically or can be forced with \\
% \author{Ben Sappey}%
% \affiliation{ 
% Authors' institution and/or address%\\This line break forced with \textbackslash\textbackslash
% }%

\date{\today}% It is always \today, today,
             %  but any date may be explicitly specified


\begin{abstract}

	Insert abstract here.

\end{abstract}

\maketitle


\onecolumngrid

\section{\label{sec:level1}Objective}

To determine the lifetime of the muon.

\section{\label{sec:level2}Introduction}

The muon was first discovered in 1936 by Carl D. Anderson and Seth Neddermeyer. They were studying cosmic radiation when Anderson noticed that certain particles curved differently from the known particles passing through a magnetic field. The negatively charged particles curved less sharply than electrons and more sharply than protons, but all carried the same velocity through the magnetic field. Originally, the charge of this particle was assumed to be of the same negative magnitude as electrons, and thus the difference in curvature was explained by giving this particle a mass greater than an electron and less than a proton. This particle was originally called a “mesotron”, the “meso” prefix meaning “middle”, as in having a mass between that of an electron or proton. Later in 1947, a particle with similar mass but dissimilar force properties was discovered. These two particles were grouped together as “mesons” instead of mesotrons (still meaning they have an intermediate mass to electrons and protons).

\section{\label{sec:level3}Apparatus and Methods}


\section{\label{sec:level4}Data Analysis}


\section{\label{sec:level5}Conclusion}


\nocite{*}
\bibliography{main}% Produces the bibliography via BibTeX.
\end{document}
